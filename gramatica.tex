\documentclass{scrartcl}
\usepackage{syntax}
\usepackage[utf8]{inputenc}

\begin{document}
\setlength{\grammarparsep}{4pt plus 1pt minus 1pt}
\begin{grammar}

<digito> ::= 1 | 2 | 3 | 4 | 5 | 6 | 7 | 8 | 9 | 0

<digitohex> ::= <digito> | a | b | c | d | e | f

<natural> ::= <digito> <natural> | <digito>

<hex> ::= <digitohex> <hex> | <digitohex>

<signo> ::= + | - |

<entero> ::= <signo> <natural> | 0x<hex>

<decimales> := . <natural> |

<exponente> ::= E+ <natural> | E- <natural> |

<real> ::= <signo> <natural> <decimales> <exponente>

<caracter> ::= a | ... | z | A | ... | Z

<booleano> ::= True | False

<op_unario> ::= ++ | -- | !

<op_binario> ::= + | / | - | * | \verb|<| | \verb|<=| | \verb|>| | \verb|>=| | \verb|==| | \verb|!=| | \verb|&| | \verb||| | \verb|^|

<Programa> ::= <Cabecera_programa> <bloque>

<bloque> ::= <Inicio_de_bloque> \\
 <Declar_de_variables_locales> \\
 <Declar_de_subprogs> \\
 <Sentencias> \\
 <Fin_de_bloque>

<Declar_de_subprogs> ::= <Declar_de_subprogs> <Declar_subprog>

<Declar_subprog> ::= <Cabecera_subprograma> <bloque>

<Declar_de_variables_locales> ::= <Marca_ini_declar_variables> \\
<Variables_locales> \\
<Marca_fin_declar_variables>

<Cabecera_programa> ::= (Dependerá del lenguaje de referencia)

<Inicio_de_bloque> ::= \{

<Fin_de_bloque> ::= \}

<Variables_locales> ::= <Variables_locales> <Cuerpo_declar_variables>
\alt <Cuerpo_declar_variables>

<Cuerpo_declar_variables> ::= (Dependerá del lenguaje de referencia)

<Cabecera_subprog> ::= (Dependerá del lenguaje de referencia)

<Sentencias> ::= <Sentencias> <Sentencia>
\alt <Sentencia>

<Sentencia> ::= <bloque>
\alt <sentencia_asignacion>
\alt <sentencia_if>
\alt <sentencia_while>
\alt <sentencia_entrada>
\alt <sentencia_salida>
\alt <sentencia_return> (si el lenguaje soporta funciones)
\alt <llamada_proced> (si el lenguaje soporta proced.)
\alt (Resto de sentencias del lenguaje asignado)

<identificador> ::= <caracter> <alfanum>

<alfanum> ::= $\varepsilon$ | <caracter> <alfanum> | <digito> <alfanum>

<sentencia_asignacion> ::=  int <identificador> = <entero>
\alt char <identificador> = <caracter>
\alt bool <identificador> = <booleano>
\alt double <identificador> = <real>

<sentencia_if> ::= if <expresion> <Inicio_de_bloque>
 <Sentencia>
 <Fin_de_bloque>

<sentencia_while> ::= while <expresion> <Inicio_de_bloque>
<Sentencia>
<Fin_de_bloque>

<sentencia_entrada> ::= <nomb_entrada> <lista_variables>

<sentencia_salida> ::= <nomb_salida> <lista_expresiones_o_cadena>

<expresion> ::= ( <expresion> )
\alt <op_unario> <expresion>
\alt <expresion> <op_binario> <expresion>
\alt <identificador>
\alt <entero>Aquí ponía constante pero en nuestro caso es lo mismo.
\alt <funcion> (si el lenguaje soporta funciones)
\alt <entero> <op_binario> <entero>
\alt <entero> <op_binario> <real>
\alt <real> <op_binario> <entero>
\alt <real> <op_binario> <real>
\alt <booleano> <op_binario> <booleano>
\alt <op_unario> <real>
\alt <op_unario> <entero>
\alt <op_unario> <booleano>
\alt (Resto de expresiones del lenguaje de referencia)


\end{grammar}
\end{document}
